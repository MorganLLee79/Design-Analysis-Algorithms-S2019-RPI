\documentclass[11pt, oneside]{article}   	% use "amsart" instead of "article" for AMSLaTeX format
\usepackage{amsfonts}
\usepackage{amsthm}
\usepackage{amsmath}
\usepackage{setspace}
\usepackage{graphicx}
\usepackage{mathtools}

\usepackage{stackengine}

\newtheorem{Question}{Question}
\newtheorem{Algorithm}{Algorithm}
\newtheorem{claim}{claim}
\graphicspath{ {images/} } %\includegraphics{name}

\usepackage{geometry}
\geometry{letterpaper, portrait, margin=1in}

\title {Design and Analysis of Algorithms Assignment 2}
\author{Harrison Lee, Alex Zhao}
\date{January 25, 2019}

\begin{document}

\maketitle

\begin{Question} (4.9) Greedy Algorithm
\end{Question}

Claim: Given two nodes, $s$ and $t$, in $n$-node undirected graph $G = (V, E)$ with a distance greater than $n/2$, there exists some node $v$ not equal to either $s$ or $t$ that, when deleted, destroys all paths from $s$ to $t$.

\begin{proof}
\begin{description}

A path of distance greater than $n/2$ takes $n/2 + 1$ nodes at least. Excluding $s$ and $t$ this is $(n-2)/2 + 1= n/2$ nodes. Let's call this path "path $A$".

For node $v$ to be deletable without destroying the other path, path $B$, $v$ cannot be in $B$.

$B$ must also have a distance greater than $n/2$ to maintain $s$ and $t$s' distance.

$B$ cannot share nodes with $A$ other than $s$ and $t$ and requires $n/2$ nodes unique from $A$.

There are only $n-2$ non-$s$ or non-$t$ nodes, but $A$ and $B$ require a total of $n$ unique nodes, so $B$ cannot exist.

\end{description}
\end{proof}

\begin{Algorithm}
\begin{description}
Find node v
\end{description}
\end{Algorithm}

Begin with a Depth-First-Search, as written in the textbook. Use it to find the shortest path from $s$ to $t$.

Mark all nodes discovered in that shortest path as "Used", numbering them by their distance to $t$.

Repeat Depth-First-Search, starting from $s$, but do not traverse past "Used" nodes. Instead, mark them as "Re-Found". Save whichever "Re-Found" node that is closest to $t$.

Once Depth-First-Search fails and cannot continue, return the saved "Re-Found" node as $v$.

\begin{proof}
\begin{description}
Prove this algorithm is $O(n+m)$:

 Depth-First-Search is known to be $O(n+m)$, as noted in the textbook. Each edge and node is traversed at most once.

This algorithm conducts Depth-First-Search twice.

This algorithm is $O(2n+2m)$, which is close enough to $O(n+m)$ for our purposes.

\end{description}
\end{proof}

\newpage

\begin{Question} (6.11) Dynamic Programming
\end{Question}
\noindent Given: You have $n$ weeks and $s_i$ parts to be produced each of those weeks. You must decide between two shipping companies, Company A which charges $r * s$ to ship in a given week while Company B charges $c$ each week in blocks of four consecutive weeks.\\

\noindent Find: A schedule deciding between company A or B for each of those $n$ weeks while following company B's restrictions. Cost is the amount paid in shipping costs. \\

\noindent Give a polynomial time algorithm that takes a sequence of supply values, $s_1, s_2, \ldots , s_n$ and returns a schedule of minimum cost.

\begin{Algorithm}
Find the optimal schedule.
\end{Algorithm}

\begin{proof}
\begin{description}

Subproblems: $OPT(j)$, or the optimal schedule from week 0 to week $j$. This can be expanded from week 4 to week n.

Recurrence: $OPT(j) = min \{\stackanchor{$r * s_j + OPT(j-1)$}{$4 * c + OPT(j-4)$}$, where $OPT(j)$ represents the optimal shipping costs possible from week 0 to week $j$. $r, s,$ and $c$ are as defined in the problem, referring to company A's cost per weight unit, total weight during week $j$, and company B's cost per week respectively.

Full Algorithm: 

OPT[0, 1, 2, 3] = [0, $s_1 * r, s_1*r + s_2 * r, s_1*r + s_2*r + s_3*r$]

for j = 0, j $\leq$ n, j++

	\quad if $OPT[j - 4] + 4*c < OPT[j-1] + r*s_j$

	\quad \quad 	$OPT[j] = OPT[j-4] + 4 * c$

	\quad else:

	\quad \quad  $OPT[j] = OPT[j-1] + r*s_j$

$return OPT[n]$

Running Time: $O(n)$. Each week is traversed once, while two already calculated $OPT(j)$s are accessed, $OPT(j-1)$ and $OPT(j-4)$, each week.

\end{description}
\end{proof}


\end{document}  






















