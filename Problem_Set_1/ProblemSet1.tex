\documentclass[11pt, oneside]{article}   	% use "amsart" instead of "article" for AMSLaTeX format
\usepackage{amsfonts}
\usepackage{amsthm}
\usepackage{amsmath}
\usepackage{setspace}
\usepackage{graphicx}
\usepackage{mathtools}

\newtheorem{Question}{Question}
\newtheorem{Algorithm}{Algorithm}
\newtheorem{claim}{claim}
\graphicspath{ {images/} } %\includegraphics{name}

\usepackage{geometry}
\geometry{letterpaper, portrait, margin=1in}

\title {Design and Analysis of Algorithms Assignment 1}
\author{Harrison Lee}
\date{\today}

\begin{document}

\maketitle

\begin{Question} %1
\begin{description}
"Bottleneck" Nodes in a Graph
\end{description}
\end{Question}

Claim: Given two nodes, $s$ and $t$, in $n$-node indirected graph $G = (V, E)$ with a distance greater than $n/2$, there exists some node $v$ not equal to either $s$ or $t$ that, when deleted, destroys all paths from $s$ to $t$.



%Claim: Every $n \in \mathbb{N} > 1$ is divisible by a prime $P_k$, where $P_k$ represents a prime number.
%$P(n): \exists P_k, m \in \mathbb{N} : n = P_k *m$ for all $n \in \mathbb{N}$

\begin{proof}
\begin{description}

A path of distance greater than $n/2$ takes $n/2 + 1$ nodes at least. Excluding $s$ and $t$ this is $(n-2)/2 + 1= n/2$ nodes. Let's call this path "path $A$".

For node $v$ to be deletable without destroying the other path, path $B$, $v$ cannot be in $B$.

$B$ must also have a distance greater than $n/2$ to maintain $s$ and $t$s' distance.

$B$ cannot share nodes with $A$ other than $s$ and $t$ and requires $n/2$ nodes unique from $A$.

There are only $n-2$ non-$s$ or non-$t$ nodes, but $A$ and $B$ require a total of $n$ unique nodes, so $B$ cannot exist.


%\item [Base Case:]$P(2): 2 = 2 * 1$, 2 is a prime number.
%Assume $P(1) \wedge P(2) \wedge \ldots \wedge P(n)$
%\item Prove $P(1) \wedge P(2) \wedge \ldots \wedge P(n) \rightarrow P(n+1)$:
%Case 1: $n+1$ is a prime number.
%Every number is divisible by itself, which in this case is a prime number. Therefore $P(n+1)$ is true.
%Case 2: $n+1 = x * y$, where $x, y < n$.
%$P(x), P(y)$ are assumed to be true, and have prime factors.
%$n+1$ has the same factors.
%Therefore, by induction, $P(n+1)$ is true $\forall n > 1$.
\end{description}
\end{proof}

\begin{Algorithm}
\begin{description}
Find node v
\end{description}
\end{Algorithm}

Begin with a Depth-First-Search, as written in the textbook. Use it to find the shortest path from $s$ to $t$.

Mark all nodes discovered in that shortest path as "Used", numbering them by their distance to $t$.

Repeat Depth-First-Search, starting from $s$, but do not traverse past "Used" nodes. Instead, mark them as "Re-Found". Save whichever "Re-Found" node that is closest to $t$.

Once Depth-First-Search fails and cannot continue, return the saved "Re-Found" node as $v$.

\begin{proof}
\begin{description}
Prove this algorithm is $O(n+m)$:

 Depth-First-Search is known to be $O(n+m)$, as noted in the textbook. Each edge and node is traversed at most once.

This algorithm conducts Depth-First-Search twice.

This algorithm is $O(2n+2m)$, which is close enough to $O(n+m)$ for our purposes.

\end{description}
\end{proof}



\begin{Question} %2
\end{Question}

Claim: $P(n): n=p_1 * p_2 * \ldots p_k \forall n > 1$, where $p$ is a prime number.

\begin{proof}
\begin{description}
\item[Base Case:] $P(2) = 2*1$

Assume $P(1) \wedge P(2) \wedge \ldots \wedge P(n)$

\item Prove $P(1) \wedge P(2) \wedge \ldots \wedge P(n) \rightarrow P(n+1)$

\underline{Case 1:} If $n+1$ is a prime number, it is divisible by itself and 1, so $P(n+1) = T$ if $n+1$ is prime.

\underline{Case 2:} Otherwise, if $n+1$ is not prime, then $n+1 = x * y$.

$x,y < n$ and $P(x), P(y) = T$, each with their own factors.

By recursively finding the factors of $P(x)$ and $P(y)$, which would then be factors to $P(n)$ as well. %INCOMPLETE, finish and format; recursive definition?

Therefore $P(n+1)$ is true, and $P(n) = T \forall n > 1$.
\end{description}
\end{proof}


\begin{Question} %3
\end{Question}

Claim: $\sum_{i=1}^n \dfrac{1}{\sqrt{i}} \geq \sqrt{n}$

\begin{proof}
\begin{description}
\item[Base Case:] $P(1): 1 \geq \sqrt{1}$

Assume $\sum_{i=1}^n \dfrac{1}{\sqrt{i}} \geq \sqrt{n}$

\item[Prove $P(n) \rightarrow P(n+1)$:] Suppose $P(n)$ is true for $n=k$. When $n=k+1$ we have.

$\sum_{i=1}^{n+1} \dfrac{1}{\sqrt{i} } \geq \sqrt{n+1}$

$\sum_{i=1}^{n} \dfrac{1}{\sqrt{i}} + \dfrac{1}{\sqrt{n+1}} \geq \sqrt{n+1}$

$\sum_{i=1}^{n} \dfrac{1}{\sqrt{i}} * \sqrt{n+1} + 1 \geq n+1$

$\sum_{i=1}^{n} \dfrac{1}{\sqrt{i}} * \sqrt{n+1} \geq n$

$\sum_{i=1}^{n} \dfrac{1}{\sqrt{i}} * \sqrt{n+1} \geq \sqrt{n} * \sqrt{n+1} \geq n$ %sanity check this

$\sqrt{n} * \sqrt{n+1} \geq \sqrt{n} * \sqrt{n}$

$\sqrt{n+1} \geq \sqrt{n}$

%$n+1 \geq n$; this is always true for $n \geq 1$.

Therefore, by induction, $P(n+1)$ is true $\forall n \geq 1$.
\end{description}
\end{proof}


\begin{Question} %4
\end{Question}

Claim: $x^n + \dfrac{1}{x^n}$ is an integer, given $x+\dfrac{1}{x}$ is an integer and $\forall n \geq 1$

\begin{proof}
\begin{description}
\item[Base Case:] $P(1): x^1 + \dfrac{1}{x^1} \in \mathbb{Z}$; $P(0): x^0 + \dfrac{1}{x^0} = 1+1 \in \mathbb{Z}$
\item Assume $x^n + \dfrac{1}{x^n} \in \mathbb{Z}$ is true for $n=1$ and $n=0$

\item[Prove $P(n)\wedge P(n-1) \wedge P(1) \rightarrow P(n+1)$:] Assume $P(n)$ and $P(n-1)$. Prove for all other cases by induction.

$x^{n+1} + \dfrac{1}{x^{n+1}}$

$ = x*x^n + \dfrac{1}{x*x^n}$
 
%$(x^n + \dfrac{1}{x^n}) * (x+\dfrac{1}{x}) = x^{n+1} + \dfrac{1}{x^{n+1}} + x^{n-1} + \dfrac{1}{x^{n-1}}$

$= (x^n + \dfrac{1}{x^n}) * (x+\dfrac{1}{x}) - (x^{n-1} + \dfrac{1}{x^{n-1}})$

Explanation: \{ $(x^n + \dfrac{1}{x^n}) * (x+\dfrac{1}{x}) = (x^{n+1} + \dfrac{1}{x^{n+1}}) + (x^{n-1} + \dfrac{1}{x^{n-1}})$ \}

Integers multiplied together or subtracted from by an integer creates another integer.

Therefore $P(n)$ is true for all $n$
\end{description}
\end{proof}


\begin{Question} %5 - impossible for two people to win over everyone else; assume everyone plays each other, and one of them wins or loses
%all connected; all wins/losses only at ends of line; if have atleast one win/loss, then linked to others
%each shape is made of earlier shapes, then connected together - internal shape will have an order, and when connecting there must be another order added; in front of or behind of
\end{Question}

 Here is a definition of a round-robin tournament.

(1) Base Case: $P(2):$ Two nodes are connected by one arrow, a win or a loss. %These two nodes can easily be lined up.

(2) Constructor Rule: When a node is added, it is given either a win or a loss with every other node.

%(3) These are the only cases contained in this function.

P(n): A network with $n$ nodes in our recursive definition can be sorted linearly so that every node lost to the node ahead of it while winning to the one before it.

\begin{proof}
\begin{description}
\item[Base Case:] P(2): The two nodes can be sorted, with the one that won in the front of the line while the other node is behind it.

\item[Prove $P(n) \rightarrow P(n+1)$:] Assume $P(n) = T$, with a working linear order of nodes. %structural induction

\underline{Case 1:} The newly added node has only wins.

It can be placed in front of all other nodes in the line. $P(n)$ will define the order of all other nodes in the line behind the new node.

\underline{Case 2:} The newly added node has only losses.

It can be placed behind all other nodes in the $P(n)$ line.

\underline{Case 3:} The new node has both wins and losses.
\iffalse
There can be at most one node with all wins and one node with all losses, as all nodes are connected.
The all wins and all losses nodes will, respectively, be at the front and back of the line.
All other nodes have both wins and losses with every other node. No node can be isolated.
Each group of nodes $k < n$ has a corresponding $P(k)$ line. Removing \fi

Temporarily ignore one of the nodes with the most losses.

The remaining nodes can create a $P(n)$ line. Pick a line where the removed node lost to the last node.

The node with the most losses can be added to the end of this line.

Therefore, by induction, $P(n) = T \forall n \in \mathbb{N}$

\end{description}
\end{proof}

\end{document}  

























